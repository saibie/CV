%%%%%%%%%%%%%%%%%%%%%%%%%%%%%%%%%%%%%%%%%
% Compact Academic CV
% LaTeX Template
% Version 2.0 (6/7/2019)
%
% This template originates from:
% https://www.LaTeXTemplates.com
%
% Authors:
% Dario Taraborelli (http://nitens.org/taraborelli/home)
% Vel (vel@LaTeXTemplates.com)
%
% License:
% CC BY-NC-SA 3.0 (http://creativecommons.org/licenses/by-nc-sa/3.0/)
%
%%%%%%%%%%%%%%%%%%%%%%%%%%%%%%%%%%%%%%%%%

%----------------------------------------------------------------------------------------
%	PACKAGES AND OTHER DOCUMENT CONFIGURATIONS
%----------------------------------------------------------------------------------------

\documentclass[11pt]{article} % Default document font size
\usepackage{kotex}

%%%%%%%%%%%%%%%%%%%%%%%%%%%%%%%%%%%%%%%%%
% Compact Academic CV
% Structural Definitions
% Version 1.0 (6/7/2019)
%
% This template originates from:
% https://www.LaTeXTemplates.com
%
% Authors:
% Dario Taraborelli (http://nitens.org/taraborelli/home)
% Vel (vel@LaTeXTemplates.com)
%
% License:
% CC BY-NC-SA 3.0 (http://creativecommons.org/licenses/by-nc-sa/3.0/)
%
%%%%%%%%%%%%%%%%%%%%%%%%%%%%%%%%%%%%%%%%%

%----------------------------------------------------------------------------------------
%	REQUIRED PACKAGES AND MISC CONFIGURATIONS
%----------------------------------------------------------------------------------------

\usepackage{graphicx} % Required for including images

\setlength{\parindent}{0pt} % Stop paragraph indentation

%----------------------------------------------------------------------------------------
%	MARGINS
%----------------------------------------------------------------------------------------
\usepackage{color}
\definecolor{sky}{RGB}{100, 100, 255}

\usepackage{geometry} % Required for adjusting page dimensions and margins

\geometry{
	paper=a4paper, % Paper size, change to letterpaper for US letter size
	top=3.25cm, % Top margin
	bottom=4cm, % Bottom margin
	left=3.5cm, % Left margin
	right=3.5cm, % Right margin
	headheight=0.75cm, % Header height
	footskip=1cm, % Space from the bottom margin to the baseline of the footer
	headsep=0.75cm, % Space from the top margin to the baseline of the header
	%showframe, % Uncomment to show how the type block is set on the page
}

%----------------------------------------------------------------------------------------
%	FONTS
%----------------------------------------------------------------------------------------

\usepackage[utf8]{inputenc} % Required for inputting international characters
\usepackage[T1]{fontenc} % Output font encoding for international characters

%\usepackage[semibold]{ebgaramond} % Use the EB Garamond font with a reduced bold weight
%\usepackage{sans}

%----------------------------------------------------------------------------------------
%	SECTION STYLING
%----------------------------------------------------------------------------------------

\usepackage{sectsty} % Allows changing the font options for sections in a document

\sectionfont{\color{sky}\fontsize{14pt}{18pt}\selectfont} % Set font options for sections
\subsectionfont{\normalsize} % Set font options for subsections
\subsubsectionfont{\mdseries\upshape\bfseries\normalsize} % Set font options for subsubsections

%----------------------------------------------------------------------------------------
%	MARGIN YEARS
%----------------------------------------------------------------------------------------

\usepackage{marginnote} % Required to output text in the margin

\newcommand{\years}[1]{\marginnote{\scriptsize #1}} % New command for adding years to the margin
\renewcommand*{\raggedleftmarginnote}{} % Left-align the years in the margin
\setlength{\marginparsep}{-10pt} % Move the margin content closer to the text
\reversemarginpar % Margin text to be output into the left margin instead of the default right margin

%----------------------------------------------------------------------------------------
%	COLOURS
%----------------------------------------------------------------------------------------

\usepackage[usenames, dvipsnames]{xcolor} % Required for specifying colours by name

%----------------------------------------------------------------------------------------
%	LINKS
%----------------------------------------------------------------------------------------

\usepackage[bookmarks, colorlinks, breaklinks]{hyperref} % Required for links

% Set link colours
\hypersetup{
	linkcolor=blue,
	citecolor=blue,
	filecolor=black,
	urlcolor=MidnightBlue
}
 % Include the file specifying the document structure and styling

% Set PDF meta-information
\hypersetup{
	pdftitle={Albert Einstein - Curriculum vitae},
	pdfauthor={Albert Einstein}
}

%----------------------------------------------------------------------------------------

\begin{document}

%----------------------------------------------------------------------------------------
%	CONTACT AND GENERAL INFORMATION
%----------------------------------------------------------------------------------------

{\LARGE\bfseries 서상협 (Seo, Sang-hyup)} % Name
\bigskip\bigskip\medskip % Whitespace

부산광역시 서구 구덕로 187\\ % Address
부산대학교병원 융합의학연구동, 49241\\
국가수리과학연구소 부산의료수학센터
\medskip % Whitespace

전\phantom{화화}화 : 051-257-8021(102)\\ % Phone number
휴대전화 : 010-4545-9199 % Mobile number
\medskip % Whitespace

E-mail: \href{mailto:saibie1677@nims.re.kr}{saibie1677@nims.re.kr}\\ % Email address
%\textsc{url}: \href{http://www.ias.edu/spfeatures/einstein/}{http://www.ias.edu/spfeatures/einstein/}\\ % Academic/personal website

\vspace{0.06\textheight} % Whitespace between contact information and specific CV information

%------------------------------------------------

Born: March 12, 1879---Ulm, Germany\\ % Date of birth
Nationality: German/American % Nationality

%------------------------------------------------

\section*{Current position}

\emph{Emeritus Professor}, Institute for Advanced Study, Princeton % Current or most recent employment position

%------------------------------------------------

\section*{Areas of specialisation}

Physics; Relativity Theory % Primary areas of research interest

%----------------------------------------------------------------------------------------
%	WORK EXPERIENCE
%----------------------------------------------------------------------------------------

\section*{Appointments held}

\years{1903-1908}Swiss Patent Office, Bern\\
\years{1908-1911}University of Bern\\
\years{1911-1912}University of Zürich\\
\years{1912-1914}Charles University of Prague\\
\years{1914-1932}Prussian Academy of Sciences, Berlin\\
\years{1920-1930}University of Leiden\\
\years{1932-1955}Institute for Advanced Study, Princeton

%----------------------------------------------------------------------------------------
%	EDUCATION
%----------------------------------------------------------------------------------------

\section*{교육}

\years{1900}\textsc{MSc} in Physics, ETH Zürich\\
\years{1900}\textsc{PhD} in Physics, ETH Zürich

%----------------------------------------------------------------------------------------
%	GRANTS, HONOURS AND AWARDS
%----------------------------------------------------------------------------------------

\section*{Grants, honours \& awards}

\years{1921}Nobel Prize in Physics, Nobel Foundation

%----------------------------------------------------------------------------------------
%	PUBLICATIONS AND TALKS
%----------------------------------------------------------------------------------------

\section*{Publications \& talks}

\subsection*{Journal articles}

\years{1901}Einstein, Albert (1901), “Folgerungen aus den Capillaritätserscheinungen (Conclusions Drawn from the Phenomena of Capillarity)", \emph{Annalen der Physik} 4: 513\\
\years{1905a}Einstein, Albert (1905), “On a Heuristic Viewpoint Concerning the Production and Transformation of Light", \emph{Annalen der Physik} 17: 132–148.\\
\years{1905b}Einstein, Albert (1905), A new determination of molecular dimensions. \emph{PhD dissertation}.\\
\years{1905c}Einstein, Albert (1905), “On the Motion—Required by the Molecular Kinetic Theory of Heat—of Small Particles Suspended in a Stationary Liquid", \emph{Annalen der Physik} 17: 549–560.\\
\years{1905d}Einstein, Albert (1905), “On the Electrodynamics of Moving Bodies", \emph{Annalen der Physik} 17: 891–921.\\
\years{1905e}Einstein, Albert (1905), “Does the Inertia of a Body Depend Upon Its Energy Content?", \emph{Annalen der Physik} 18: 639–641.\\
\years{1915}Einstein, Albert (1915), “Die Feldgleichungen der Gravitation (The Field Equations of Gravitation)", \emph{Koniglich Preussische Akademie der Wissenschaften}: 844–847\\
\years{1917a}Einstein, Albert (1917), “Kosmologische Betrachtungen zur allgemeinen Relativitätstheorie (Cosmological Considerations in the General Theory of Relativity)", \emph{Koniglich Preussische Akademie der Wissenschaften}\\
\years{1917b}Einstein, Albert (1917), “Zur Quantentheorie der Strahlung (On the Quantum Mechanics of Radiation)", \emph{Physikalische Zeitschrift} 18: 121–128

%------------------------------------------------

\subsection*{Books}

\years{1954}Einstein, Albert (1954), \emph{Ideas and Opinions}, New York: Random House, ISBN 0-517-00393-7

%------------------------------------------------

\subsection*{Newspaper articles}

\years{1940}Einstein, Albert, et al. (December 4, 1948), “To the editors", \emph{New York Times}\\
\years{1949}Einstein, Albert (May 1949), “Why Socialism?", \emph{Monthly Review}.

%----------------------------------------------------------------------------------------
%	TEACHING
%----------------------------------------------------------------------------------------

\section*{Teaching}

\ldots

%------------------------------------------------

\section*{Service to the profession}

\ldots

\vfill % Whitespace before final footer

%----------------------------------------------------------------------------------------
%	FINAL FOOTER
%----------------------------------------------------------------------------------------

% Any final footer text such as a URL to the latest version of this CV, last updated date, compiled in XeTeX, etc
\begin{center}
	\scriptsize
	Last updated: \today~~\raisebox{-0.5pt}{\textbullet}~~\href{https://www.LaTeXTemplates.com}{https://www.LaTeXTemplates.com}
\end{center}

%----------------------------------------------------------------------------------------

\end{document}
